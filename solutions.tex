\chapter{Exercise Solutions}
\stoptocwriting

\setcounter{section}{3}
\section*{Chapter 3}
\begin{enumerate}
	\item A "bundle" refers to a series of signal which are related to each other. It is often referred to as a "bus", but is referred to by a different name to prevent confusion.
	
	\item A bundle is often shown in black box diagrams as a wire with a slash on it and a number above it indicating the number of signals contained in the bundle.
	
	\item It is considered good practice to draw a black-box diagram of components before implementing them in VHDL as this helps provide a visual representation of each component. In addition, it helps eliminate confusion from having many logic gates inside and allows for the reuse of components elsewhere in your code.
	
	\item The given black-box drawings can be represented with the following VHDL code.
	
	\noindent
	\begin{minipage}{1\linewidth}
		a)
		\begin{lstlisting}[]		
ENTITY sys1 IS
	PORT (
		a_in1		:	IN	STD_LOGIC;
		b_in2		:	IN	STD_LOGIC;
		clk			:	IN	STD_LOGIC;
		ctrl_int	:	IN	STD_LOGIC;
		out_b		:	OUT STD_LOGIC);
END sys1;
		\end{lstlisting}
	\end{minipage}

	\begin{minipage}{1\linewidth}
		b)
		\begin{lstlisting}[]
ENTITY sys2 IS
	PORT (
		input_w	:	IN	STD_LOGIC;
		a_data	:	IN	STD_LOGIC_VECTOR(7 DOWNTO 0);
		b_data	:	IN	STD_LOGIC_VECTOR(7 DOWNTO 0);
		clk		:	IN	STD_LOGIC;
		dat_4	:	OUT	STD_LOGIC_VECTOR(7 DOWNTO 0);
		dat_5	:	OUT	STD_LOGIC_VECTOR(2 DOWNTO 0));
END sys2;
		\end{lstlisting}
	\end{minipage}

	\item The black-box figures are shown below.
	
	\begin{minipage}{1\linewidth}
		\vspace{5pt}
		a)\\
		\begin{tikzpicture}[x=1mm,y=1mm,line width=0.8pt,scale=0.8,framed]
		%\draw[help lines] (0,0) grid (50,50);
		% BOX
		\draw (20,-5) rectangle (37,25) node[midway]{ckt\_c};
		% INPUTS
		\small
		\node (hide) at (0,30) {}; % just to expand background
		\node (a) at (20,-2.5) {}; % this is the reference point
		\draw [latex-] ($(a)+(0,25)$) -- ++(-10,0) node[left]{bun\_a}
		node[pos=0.4,above]{8} node[pos=0.7]{/};
		\draw [latex-] ($(a)+(0,20)$) -- ++(-10,0) node[left]{bun\_b}
		node[pos=0.4,above]{8} node[pos=0.7]{/};
		\draw [latex-] ($(a)+(0,15)$) -- ++(-10,0) node[left]{bun\_c}
		node[pos=0.4,above]{8} node[pos=0.7]{/};
		\draw [latex-] ($(a)+(0,10)$) -- ++(-10,0) node[left]{lda};
		\draw [latex-] ($(a)+(0,5)$) -- ++(-10,0) node[left]{ldb};
		\draw [latex-] ($(a)+(0,0)$) -- ++(-10,0) node[left]{ldc};
		% OUTPUTS
		\draw [-latex] ($(a)+(17,19)$) -- ++(10,0) node[right]{reg\_a} node[pos=0.6,above]{8} node[pos=0.4]{/};
		\draw [-latex] ($(a)+(17,12)$) -- ++(10,0) node[right]{reg\_b} node[pos=0.6,above]{8} node[pos=0.4]{/};
		\draw [-latex] ($(a)+(17,5)$) -- ++(10,0) node[right]{reg\_c} node[pos=0.6,above]{8} node[pos=0.4]{/};
		\end{tikzpicture}
	\end{minipage}

	\begin{minipage}{1\linewidth}
		\vspace{5pt}
		b)\\
		\begin{tikzpicture}[x=1mm,y=1mm,line width=0.8pt,scale=0.8,framed]
		%\draw[help lines] (0,0) grid (50,50);
		% BOX
		\draw (20,-10) rectangle (37,25) node[midway]{ckt\_e};
		% INPUTS
		\small
		\node (hide) at (0,30) {}; % just to expand background
		\node (a) at (20,-2.5) {}; % this is the reference point
		\draw [latex-] ($(a)+(0,25)$) -- ++(-10,0) node[left]{RAM\_CS};
		\draw [latex-] ($(a)+(0,20)$) -- ++(-10,0) node[left]{RAM\_WE};
		\draw [latex-] ($(a)+(0,15)$) -- ++(-10,0) node[left]{RAM\_OE};
		\draw [latex-] ($(a)+(0,10)$) -- ++(-10,0) node[left]{SEL\_OP1}
		node[pos=0.4,above]{4} node[pos=0.7]{/};
		\draw [latex-] ($(a)+(0,5)$) -- ++(-10,0) node[left]{SEL\_OP2}
		node[pos=0.4,above]{4} node[pos=0.7]{/};
		\draw [latex-] ($(a)+(0,0)$) -- ++(-10,0) node[left]{RAM\_DATA\_IN}
		node[pos=0.4,above]{8} node[pos=0.7]{/};
		\draw [latex-] ($(a)+(0,-5)$) -- ++(-10,0) node[left]{RAM\_ADDR\_IN}
		node[pos=0.4,above]{10} node[pos=0.7]{/};
		% OUTPUTS
		\draw [-latex] ($(a)+(17,10)$) -- ++(10,0) node[right]{RAM\_DATA\_OUT} node[pos=0.6,above]{8} node[pos=0.4]{/};
		\end{tikzpicture}
	\end{minipage}

	\item a) Line 4 is missing a semicolon, and the semicolon in line 5 should be outside of the parenthesis. Correction:
	
	\begin{minipage}{1\linewidth}
		\begin{lstlisting}[]
ENTITY ckt_a IS
	PORT (
		J,K	:	IN	STD_LOGIC;
		CLK	:	IN	STD_LOGIC;
		Q	:	OUT	STD_LOGIC);
END ckt_a;
		\end{lstlisting}
	\end{minipage}
		
	b) Line 5 should have double parentheses before the semicolon. \mbox{Correction:}
	
	\begin{minipage}{1\linewidth}
		\begin{lstlisting}[]
ENTITY ckt_b IS
	PORT (
		mr_fluffy	:	IN	STD_LOGIC_VECTOR(15 DOWNTO 0);
		mux_ctrl	:	IN	STD_LOGIC_VECTOR(3 DOWNTO 0);
		byte_out	:	OUT	STD_LOGIC_VECTOR(3 DOWNTO 0));
END ckt_b;
		\end{lstlisting}
	\end{minipage}
	
\end{enumerate}
\resumetocwriting